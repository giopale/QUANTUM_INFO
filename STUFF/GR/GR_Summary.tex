\documentclass[a4paper]{article}

%% Language and font encodings
\usepackage[english]{babel}
\usepackage[utf8x]{inputenc}
\usepackage[T1]{fontenc}

%% Sets page size and margins
\usepackage[a4paper,top=3cm,bottom=2cm,left=2.7cm,right=2.7cm,marginparwidth=1.75cm]{geometry}

%% Useful packages
\usepackage{amsmath}
\usepackage{amsfonts}
\usepackage{bm}
\usepackage{graphicx}
\usepackage[colorinlistoftodos]{todonotes}
\usepackage[colorlinks=true, all colors=blue]{hyperref} %referenze linkate
\usepackage{booktabs}
\usepackage{siunitx}  %notaz. espon. con \num{} e unità di misura in SI con \si{}
\usepackage{xcolor}
\usepackage{colortbl}
\usepackage{bm}
\usepackage{caption} 
\usepackage{indentfirst}
\usepackage{physics} 
\usepackage{rotating}
\usepackage{tabularx}
\usepackage{url}
\usepackage{pst-plot}
\usepackage{comment} %per usare l'ambiente {comment}
\usepackage{float} 
\usepackage{subfig}
\usepackage[americanvoltages]{circuitikz} %per disegnare circuiti
\usepackage{tikz}
\usepackage{mathtools} %per allineare su più linee in ambiente {align} o {align*}
\usepackage{cancel}
\usepackage{listings}
\renewcommand{\CancelColor}{\color{lightgray}}
%\setlength{\parindent}{0cm}


%%%%%%%%%% HEADERS AND FOOTERS %%%%%%%%%%%%
\newcommand{\theexercise}{Ex. 2}
\newcommand{\thedate}{October 20, 2020}
\usepackage{fancyhdr}

\pagestyle{fancy}
\fancyhf{}
% \lhead{Giorgio Palermo}
% \rhead{\thedate}
\lfoot{General Relativity 20/21}
% \cfoot{\theexercise}
\rfoot{Page \thepage}


\graphicspath{{Figure/}}
\captionsetup{format=hang,labelfont={sf,bf},font=small}
\captionsetup{tableposition=top,figureposition=bottom,font=small}
\captionsetup[table]{skip=8pt}

\renewcommand{\phi}{\varphi}





\begin{document}
\hypersetup{linkcolor = black}
\hypersetup{linkcolor = blue}
\thispagestyle{plain}
\author{Checco, Nenni, Giopale}
\title{GR Summary}
\maketitle
% \begin{center}
%     \textbf{MASTER'S DEGREE IN PHYSICS}
    
%     Academic Year 2020-2021
    
%     \medskip
%     \textbf{STRONGLY CORRELATED SYSTEMS}
% \end{center}

% \vspace{0.0cm}
% Student: Giorgio Palermo

% Student ID: 1238258

% Date: \thedate
% \begin{center}
% \textbf{STUFF TO REMEMBER}
% \medskip
% \end{center}
\noindent
\begin{center}
\textit{Remember, remember the 5th of November...}
\end{center}
\subsection*{Precessione del perielio}

Appunti riguardo al \textit{Problem 3} dell' HW6:
\begin{enumerate}
  \item Trovare l'equazione che regola il moto di una particella massiva nella metrica di Schwarzschild 
  \item Trovare lo scostamento angolare del perielio per orbita.
\end{enumerate}

Primo punto: cerchiamo una equazione per il moto della particella nella metrica di Scwarzschild
\[
\dd{s}^2 = -\pqty{1-\frac{2GM}{r}} \dd{t}^2 + \pqty{1-\frac{2GM}{r}}^{-1} \dd{r}^2 + r^2\pqty{\dd{\theta}^2 + \sin^2(\varphi)\dd{\varphi^2}}
\]
Sfruttiamo i due killing vectors per trovare delle quantit\`a conservate:
\begin{align*}
\xi^\mu = (1,0,0,0) &\implies e= -\xi^\mu u_\mu = \text{const} = \pqty{1-\frac{2GM}{r}}\dv{t}{\tau} \\
\xi^\mu = (0,0,0,1) &\implies l= \xi^\mu u_\mu = \text{const} = r^2 \dv{\phi}{\tau} 
\end{align*}
Poi facciamo qualcosa di inaspettato: sfruttiamo la normalizzazione di $u^\mu:$\[ -1=u^\mu u_\mu  = g_{\alpha\beta}u^\alpha u^\beta\] e dopo qualche passaggio troviamo:
\begin{equation}
\frac{1}{2}\pqty{\dv{r}{\tau}}^2 = \bqty{-\frac{GM}{r} + \frac{l^2}{2r^2}-\frac{GMl^2}{r^3}} = \frac{e^2 -1}{2}.\label{eq:moto_particella}
\end{equation}
Ci ricordiamo che una traiettoria \`e una relazione $r\leftrightarrow\phi$ e quindi troviamo un modo per eliminare la $\tau$ dall'equazione; in particolare usiamo la conservazione di $l:$ \[l=r^2\dv{\phi}{\tau} \implies\dv{}{\tau} = \dv{\phi}{\tau}\dv{}{\phi} = \frac{l}{r^2}\dv{}{\phi}\]
Definiamo per comodit\`a anche la variabile $u=r^{-1}$ trasformiamo la derivata rispetto a $\phi:$\[u=\frac{1}{r} \rightarrow \dv{r}{\phi} =\dv{r}{u}\dv{u}{\phi} = -\frac{1}{u^2}\dv{u}{\phi}. \]
L'equazione del moto dopo un massaggino diventa:
\begin{equation} \dv[2]{u}{\phi} + u = \frac{GM}{l^2}+3GMu^2 \label{eq:moto_particella_1}\end{equation} che \`e quello che ci era chiesto di confermare dal primo punto.

Secondo punto: trovare la precessione del perielio per ogni orbita. Per far questo consideriamo
\begin{enumerate}
  \item $u_c = \frac{GM}{l^2}+3GMu_c^2$ orbite circolari
  \item $u(\phi)=u_c\bqty{1+w(\phi)}$ linearizzazione della soluzione per $u\sim u_c$
\end{enumerate}
Sostituiamo in \ref{eq:moto_particella_1} e otteniamo un oscillatore armonico:
\[\dv[2]{w}{\phi} + \pqty{1-6GMu_c}w=0 \rightarrow \Omega = \sqrt{1-6GMu_c} \]
che ci d\`a un periodo di:\[\Delta\Phi_{orb}=\frac{2\pi}{\omega} \sim 2\pi +6GM u_c\] al quale possiamo sottrarre $2\pi$ e sostituire il valore del raggio per l'orbita circolare $r_c=l^2/GM$ per ottenere la differenza di fase per precessione:\[ \delta\phi_{prec}=6\pi\pqty{\frac{GM}{l}}^2.\]
... and that's all folks!






\end{document}
