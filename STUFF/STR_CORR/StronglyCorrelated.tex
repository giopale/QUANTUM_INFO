\documentclass[a4paper]{article}

%% Language and font encodings
\usepackage[english]{babel}
\usepackage[utf8x]{inputenc}
\usepackage[T1]{fontenc}

%% Sets page size and margins
\usepackage[a4paper,top=3cm,bottom=2cm,left=2.7cm,right=2.7cm,marginparwidth=1.75cm]{geometry}

%% Useful packages
\usepackage{amsmath}
\usepackage{amsfonts}
\usepackage{bm}
\usepackage{graphicx}
\usepackage[colorinlistoftodos]{todonotes}
\usepackage[colorlinks=true, all colors=blue]{hyperref} %referenze linkate
\usepackage{booktabs}
\usepackage{siunitx}  %notaz. espon. con \num{} e unità di misura in SI con \si{}
\usepackage{xcolor}
\usepackage{colortbl}
\usepackage{bm}
\usepackage{caption} 
\usepackage{indentfirst}
\usepackage{physics} 
\usepackage{rotating}
\usepackage{tabularx}
\usepackage{url}
\usepackage{pst-plot}
\usepackage{comment} %per usare l'ambiente {comment}
\usepackage{float} 
\usepackage{subfig}
\usepackage[americanvoltages]{circuitikz} %per disegnare circuiti
\usepackage{tikz}
\usepackage{mathtools} %per allineare su più linee in ambiente {align} o {align*}
\usepackage{cancel}
\usepackage{listings}
\renewcommand{\CancelColor}{\color{lightgray}}
%\setlength{\parindent}{0cm}


%%%%%%%%%% HEADERS AND FOOTERS %%%%%%%%%%%%
\newcommand{\theexercise}{Ex. 2}
\newcommand{\thedate}{October 20, 2020}
\usepackage{fancyhdr}

\pagestyle{fancy}
\fancyhf{}
\lhead{Giorgio Palermo}
\rhead{\thedate}
\lfoot{Quantum Information 20/21}
\cfoot{\theexercise}
\rfoot{Page \thepage}

%%%%%%%%%% CODE LISTING %%%%%%%%%%%
%New colors 
\definecolor{codegreen}{HTML}{92c42a}
\definecolor{codegray}{rgb}{0.5,0.5,0.5}
\definecolor{codepurple}{HTML}{f92472}
\definecolor{codeblue}{HTML}{67d8ef}
\definecolor{codeyellow}{HTML}{e68f29}%{e4ab24}
\definecolor{codemagenta}{HTML}{f92472}
\definecolor{backcolour}{rgb}{0.95,0.95,0.92}


%Code listing style named "mystyle"
\lstdefinestyle{mystyle}{
  language={[03]Fortran},
  backgroundcolor=\color{backcolour},   commentstyle=\color{codegray},
  keywordstyle=\color{codemagenta},
  numberstyle=\tiny\color{codegray},
  stringstyle=\color{codeyellow},
  basicstyle=\ttfamily\footnotesize,
  breakatwhitespace=false,         
  breaklines=true,                 
  captionpos=b,                    
  keepspaces=true,                 
  numbers=left,                    
  numbersep=5pt,                  
  showspaces=false,                
  showstringspaces=false,
  showtabs=false,                  
  tabsize=2
}
%"mystyle" code listing set
\lstset{style=mystyle}


\graphicspath{{Figure/}}
\captionsetup{format=hang,labelfont={sf,bf},font=small}
\captionsetup{tableposition=top,figureposition=bottom,font=small}
\captionsetup[table]{skip=8pt}







\begin{document}
\hypersetup{linkcolor = black}
\hypersetup{linkcolor = blue}
\thispagestyle{plain}
\begin{center}
    \textbf{MASTER'S DEGREE IN PHYSICS}
    
    Academic Year 2020-2021
    
    \medskip
    \textbf{STRONGLY CORRELATED SYSTEMS}
\end{center}

\vspace{0.0cm}
Student: Giorgio Palermo

Student ID: 1238258

Date: \thedate
\begin{center}
\textbf{STUFF TO REMEMBER}
\medskip
\end{center}
\noindent
\textit{In this file I will collect some elements from the class by prof. dell'Anna.}

\section*{Path Integrals}

Schroedinger Equation is:
\begin{equation}
  i\hbar\pdv{t}\ket{\psi}=H(\va{p}, \va{x})\ket{\psi}
\end{equation}
with solution
\begin{equation}
  \ket{\psi(t')}= e^{\frac{i}{\hbar}}H(\va{x},\va{p})(t'-t)\ket{\psi(t)}
\end{equation}
Coordinate representation:
\begin{align}
  \braket{x'}{\psi(t')}&=\mel{x'}{U(t',t)}{\psi(t)}\\
& =\int \dd{\va{x}}\mel{x'}{U(t',t)}{x}\braket{x}{\psi(t)}\\
&= \int \dd{\va{x}} U(x',t',x,t)\psi(x,t)
\end{align}
The generic matrix element of U is:
\begin{equation}
  U(\va{x}_f, t_f,\va{x}_i, t_i)= \mel{\va{x}_f}{U(t_f,t_i)}{\va{x}_i} = \mel{\va{x}_f}{e^{\frac{-i}{\hbar}H(\va{p},\va{x})(t'-t)}}{\va{x}_i}
\end{equation}
Splitting the time interval into $M$ small pieces and sending $m\to\infty$ one is able (after some long calculations) to write $U$ as:
\begin{equation}
  U(\va{x}_f,t_f,\va{x}_i,t_i) = \int \mathcal{D}\bqty{\va{x}(t)}\mathcal{D}\bqty{p(t)}e^{\frac{i}{\hbar}\int_{t_i}^{t_f}\dd{t}\bqty{p(t)\dot{\va{x}}(t) - H(\va{x},\va{p})}}
\end{equation}
or in the equivalent form
\begin{equation}
  U(\va{x}_f,t_f,\va{x}_i,t_i) = \int \mathcal{D}\bqty{\va{x}(t)}e^{\frac{i}{\hbar}\int_{t_i}^{t_f}\dd{t}\bqty{\frac{m}{2}\pqty{\dv{\va{x}}{t}}^2 - V(\va{x}(t))}}
\end{equation}
the analogous for \textbf{statistical mechanics} is derived from the previous using a \textbf{Wick's rotation} $t=-i\tau:$
\begin{equation}
  U(\va{x}_f,t_f,\va{x}_i,t_i) = \int \mathcal{D}\bqty{\va{x}(\tau)}e^{-\frac{1}{\hbar}\int_{\tau_i}^{\tau_f}\dd{\tau}\bqty{\frac{m}{2}\pqty{\dv{\va{x}}{\tau}}^2 - V(\va{x}(\tau))}}
\end{equation}
\textbf{Partition  function} in statistical mechanics is defined as $\Tr(e^{-\beta H}),$ which is
\begin{align}
  Z = \Tr(e^{-\beta H}) &= \int \dd{\va{x}}\ev{e^{-\beta H}}{x}\\
&= \int \dd{\va{x}}\underset{\va{x}(0) = \va{x}(\beta\hbar)}{\int} \mathcal{D}\bqty{\va{x}(\tau)}e^{-\frac{1}{\hbar}\int_{0}^{\beta\hbar}\dd{\tau}\bqty{\frac{m}{2}\pqty{\dv{\va{x}}{\tau}}^2 - V(\va{x}(\tau))}}
 \end{align}






\end{document}
